% ==================================================================
%  macros.tex – Commandes & environnements personnalisés
% ==================================================================

% ------------------------------------------------------------------
% Titres plus élégants
% ------------------------------------------------------------------
\titleformat{\chapter}[display]
  {\normalfont\LARGE\bfseries\color{DarkBlue}}
  {Chapitre \thechapter}{1ex}{}
  {\Huge\bfseries}
\titlespacing*{\chapter}{0pt}{-0.5cm}{1cm}

\titleformat{\section}
  {\normalfont\Large\bfseries\color{DarkBlue}}
  {\thesection}{0.75em}{}

% ------------------------------------------------------------------
% En-têtes / pieds de page
% ------------------------------------------------------------------
\pagestyle{fancy}
\fancyhf{}
\fancyhead[LE,RO]{\thepage}
\fancyhead[LO]{\nouppercase{\rightmark}}
\fancyhead[RE]{\nouppercase{\leftmark}}
\renewcommand{\headrulewidth}{0.4pt}
\renewcommand{\footrulewidth}{0pt}

% ------------------------------------------------------------------
% Environnements pédagogiques
% ------------------------------------------------------------------
% ----- Compteur ------------------------------------------------------
% =====================================================================
%  Exercice
% =====================================================================
\newcounter{exercice}[chapter]

\newtcolorbox{exercicebox}[2][]{
  enhanced,               % active les calques titre/corps
  breakable,              % coupure de page autorisée
  sharp corners,
  colframe=DarkBlue,
  colback=LightGray!50,
  colbacktitle=DarkBlue!15,
  coltitle=DarkBlue!90!black,
  fonttitle=\bfseries,
  title={Exercice~\thechapter.\theexercice%
         \ifstrempty{#2}{}{\;--\;#2}},
  #1,                     % options supplémentaires éventuelles
  before skip=1em, after skip=1em,
  left=2mm, right=2mm, top=1mm, bottom=1mm,
}

\newenvironment{exercice}[1][]%
  {\refstepcounter{exercice}\begin{exercicebox}{#1}}%
  {\end{exercicebox}}

% =====================================================================
%  Travail pratique (TP)
% =====================================================================
\newcounter{tp}[chapter]

\newtcolorbox{tpbox}[2][]{
  enhanced,
  breakable,
  sharp corners,
  colframe=Maroon,
  colback=LightGray!40,
  colbacktitle=Maroon!15,
  coltitle=Maroon!90!black,
  fonttitle=\bfseries,
  title={Travail~pratique~\thechapter.\thetp%
         \ifstrempty{#2}{}{\;--\;#2}},
  #1,
  before skip=1em, after skip=1em,
  left=2mm, right=2mm, top=1mm, bottom=1mm,
}

\newenvironment{tp}[1][]%
  {\refstepcounter{tp}\begin{tpbox}{#1}}%
  {\end{tpbox}}

% Environnement "réflexion"

\newtcolorbox{reflexion}{
  colback=ForestGreen!10,
  colframe=ForestGreen,
  title=\faLightbulbO~Question de réflexion,
  fonttitle=\bfseries,
  before skip=1em, after skip=1em
}

% ------------------------------------------------------------------
% Raccourcis mathématiques fréquents
% ------------------------------------------------------------------
\newcommand{\bigO}{\mathcal{O}}
\newcommand{\NP}{\textsf{NP}}
\newcommand{\Pclass}{\textsf{P}}
\newcommand{\Z}{\mathbb{Z}}
\newcommand{\R}{\mathbb{R}}

% ------------------------------------------------------------------
% Inclusion simplifiée de listings
% ------------------------------------------------------------------
\newcommand{\codefile}[2][]{%
  \lstinputlisting[caption={\detokenize{#2}},#1]{#2}%
}

% ------------------------------------------------------------------
% Icônes FontAwesome pour pimenter (optionnel)
% ------------------------------------------------------------------
\usepackage{fontawesome}

