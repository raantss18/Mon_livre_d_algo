% ==================================================================
%  packages.tex – Préambule partagé pour le livre d’algorithmique
%  Compatible pdfLaTeX ; bascule automatiquement vers Xe/Lua si dispo
% ==================================================================

% ------------------------------------------------------------------
% Encodage & langue
% ------------------------------------------------------------------
\usepackage[utf8]{inputenc}   % Encodage source (pdfLaTeX)
\usepackage[T1]{fontenc}      % Codage des fontes
\usepackage[french]{babel}    % Césure & typographie FR
\usepackage{csquotes}         % Guillemets intelligents
\usepackage{microtype}        % Micro-typographie (protrusion, expansion)

% ------------------------------------------------------------------
% Polices modernes lisibles
% ------------------------------------------------------------------
\usepackage{lmodern}          % Texte (Latin Modern)
\usepackage[scaled]{inconsolata} % Monospace (Inconsolata) – listings
\usepackage{newtxmath,newtxtext}  % Math joli + texte plus compact

% ------------------------------------------------------------------
% Mise en page & hyperliens
% ------------------------------------------------------------------
\usepackage[a4paper,margin=2.5cm]{geometry}
\usepackage{fancyhdr}         % En-têtes/pieds de page
\usepackage{titlesec}         % Personnalisation titres
\usepackage{tocloft}          % TOC raffiné
\usepackage{enumitem}         % Listes compactes
\usepackage{hyperref}
\hypersetup{
  colorlinks=true,
  linkcolor=Maroon,
  citecolor=ForestGreen,
  urlcolor=RoyalBlue,
  pdfstartview=FitH,
  pdfauthor={Antsa Raniriamanjaka},
  pdftitle={Algorithmique – Cours, Exercices, TP}
}

% ------------------------------------------------------------------
% Graphiques, couleurs, TikZ
% ------------------------------------------------------------------
\usepackage{xcolor}
\definecolor{DarkBlue}{HTML}{002B63}
\definecolor{RoyalBlue}{HTML}{0066CC}
\definecolor{ForestGreen}{HTML}{228833}
\definecolor{Maroon}{HTML}{882255}
\definecolor{LightGray}{HTML}{F2F2F2}

\usepackage{etoolbox}   % Tests sur arguments vides


\usepackage{graphicx}
\usepackage{tikz}
\usetikzlibrary{arrows.meta,positioning,calc,graphs}

% ------------------------------------------------------------------
% Code & algorithmes
% ------------------------------------------------------------------
\usepackage{listings}
\lstdefinestyle{algocpp}{
  language=C++,
  basicstyle=\small\ttfamily,
  keywordstyle=\color{RoyalBlue}\bfseries,
  commentstyle=\color{ForestGreen}\itshape,
  stringstyle=\color{Maroon},
  numbers=left, numberstyle=\tiny, stepnumber=1,
  backgroundcolor=\color{LightGray},
  frame=single, framerule=0pt,
  tabsize=2, showstringspaces=false,
  captionpos=b
}
\lstset{
  inputencoding=utf8,
  extendedchars=true,   % garde les caractères 8-bits
  literate=
    {←}{{$\leftarrow$}}1
    {→}{{$\rightarrow$}}1
    {≠}{{$\neq$}}1
    {≥}{{$\ge$}}1
    {≤}{{$\le$}}1
    {×}{{$\times$}}1
    {÷}{{$\div$}}1
}

\lstset{style=algocpp}

% ------------------------------------------------------------------
% Boîtes colorées pour TP/Exercices
% ------------------------------------------------------------------
\usepackage[skins,breakable,most]{tcolorbox}
\tcbset{
  enhanced,
  sharp corners=all,
  boxrule=0pt,
  colback=LightGray,
  colframe=DarkBlue,
  coltitle=DarkBlue,
  fonttitle=\bfseries\large,
  before skip=1em, after skip=1em
}

% ------------------------------------------------------------------
% Réglages diverses utiles
% ------------------------------------------------------------------
\setcounter{tocdepth}{2}      % Jusqu'au niveau \section dans la TOC
\setlength{\parindent}{0pt}
\setlength{\parskip}{0.6em}
\renewcommand{\baselinestretch}{1.08}
