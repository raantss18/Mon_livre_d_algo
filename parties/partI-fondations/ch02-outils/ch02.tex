% ======================================================================
%  Chapitre 2 – Outils mathématiques pour l’analyse d’algorithmes
%  Version enrichie : définitions + exemples détaillés destinés à des
%  débutantes / débutants du premier cycle universitaire.
%  Références principales : Cormen et al. (CLRS) ch. 2–4 ; Kleinberg &
%  Tardos chap. 0 ; Sedgewick & Wayne chap. 1 ; Skiena « War Stories ».
% ======================================================================

\chapter{Outils mathématiques pour l’analyse}

% ----------------------------------------------------------------------
% Épigraphe motivante
% ----------------------------------------------------------------------
\begin{flushright}\small
« Les mathématiques sont la grammaire de la science. »\\[-0.2em]
— \textit{Carl Friedrich Gauss}
\end{flushright}

\section*{Objectifs du chapitre}
\begin{itemize}[label=\small$\blacktriangleright$]
  \item Poser les briques mathématiques indispensables : sommes, suites,
        logarithmes.
  \item Comprendre et \emph{utiliser} les notations asymptotiques
        ($\bigO$, $\Theta$, $\Omega$).
  \item Savoir démontrer une propriété par induction \textbf{et} savoir
        l’appliquer à un algorithme concret.
  \item Acquérir une méthode pour résoudre les relations de récurrence
        les plus fréquentes (dichotomie, tri fusion, etc.).
\end{itemize}
\vspace{0.8em}

% **********************************************************************
\section{Pourquoi des maths dans l’informatique ?}
Un algorithme n’est pas seulement un bout de code ; c’est une idée
\emph{prouvable}. Les mathématiques servent :
\begin{enumerate}
  \item à \textbf{décrire} précisément le problème et l’algorithme ;
  \item à \textbf{démontrer} sa correction (le résultat est toujours bon) ;
  \item à \textbf{quantifier} ses ressources (temps, mémoire, bande
        passante, énergie).\footnote{Voir l’introduction de
        \cite{Cormen2022} pour une discussion classique.}
\end{enumerate}

\begin{reflexion}
Pensez à un jeu vidéo, une application bancaire et un réseau social. Où, selon
vous, se cachent les calculs mathématiques ? (Indices : trajectoires de
personnages, chiffrement RSA / ECC, moteurs de recommandation.)
\end{reflexion}

% **********************************************************************
\section{Sommes, suites et notations de base}

\subsection{Suites numériques}
Une \emph{suite} $(u_n)_{n\ge 0}$ est une fonction de $\mathbb N$ vers
$\mathbb R$. Elle peut être définie :
\begin{itemize}
  \item \textbf{explicitement} : $u_n = 3n+2$ ;
  \item \textbf{récursivement} : $u_{n+1}=2u_n+1$ avec $u_0=0$.
\end{itemize}
En pratique, on utilise une suite pour \textbf{compter} combien d’opérations $T(n)$ sont nécessaires lorsqu’on agrandit la taille de l’entrée $n$. Ainsi, chaque terme de la suite montre « le coût quand on ajoute un élément de plus ».


\subsection{Sommes arithmétiques et géométriques}
\begin{align*}
\sum_{i=1}^{n} i &\;=\; \frac{n(n+1)}{2} && \text{(Gauss : addition « en miroir »)}\\[0.4em]
\sum_{i=0}^{n-1} r^{i} &\;=\; \frac{1-r^{n}}{1-r}\quad (r\neq 1) && \text{(raison $r$ ; géométrique)}
\end{align*}
\emph{Application} : si un tri naïf effectue $(n-1)+\dots+2+1$ comparaisons,
il exécute $\tfrac{n(n-1)}{2}\in\Theta(n^2)$ opérations.

\subsection{Logarithmes — rappels essentiels}
Soient $a$ et $b$ deux réels strictement positifs :\\
\begin{itemize}
  \item $\log_b a$ = exposant $x$ tel que $b^x=a$.
  \item Bases usuelles : $2$ (binaire), $10$ (décimale), $e$ (naturel).
  \item \textbf{Changement de base} : $\log_b a = \dfrac{\log_k a}{\log_k b}$.
  \item \textbf{Croissance lente} : $\log n\ll n^{\varepsilon}$ pour tout
        $\varepsilon>0$.
\end{itemize}

\begin{exercice}[Manipuler les logarithmes]
Montrez que $\displaystyle\log_2(n!)\in\Theta(n\log n)$.
\textit{Indication :} utilisez la somme $\sum_{k=1}^{n} \log_2 k$ et comparez‑la
à une intégrale.
\end{exercice}

% **********************************************************************
\section{Principe d’induction : mode d’emploi}

\subsection{Induction simple (ou faible)}
\begin{enumerate}
  \item \textbf{Base} : vérifier $P(n_0)$.
  \item \textbf{Hérédité} : supposer $P(k)$ vraie et montrer $P(k+1)$.
\end{enumerate}
Si les deux étapes tiennent, $P(n)$ est vraie $\forall n\ge n_0$.

\subsection{Induction forte}
Hypothèse plus large : on suppose $P(n_0),\ldots,P(k)$ pour démontrer $P(k+1)$.
Utile quand la valeur courante dépend de \emph{plusieurs} valeurs précédentes
(ex. suite de Fibonacci).

%\paragraph{Exemple détaillé — somme des $n$ premiers entiers}
\begin{exercice}[Somme des premiers entiers]
Prouver que $S(n)=\sum_{i=1}^{n} i = \frac{n(n+1)}{2}$.
\end{exercice}
\textbf{Solution}
\begin{itemize}
  \item \emph{Base ($n=1$)} : $S(1)=1$ et $\tfrac{1(1+1)}{2}=1$ ; ok.
  \item \emph{Induction} : supposons la formule vraie pour $n=k$.
        Alors $S(k+1)=S(k)+(k+1)=\frac{k(k+1)}{2}+k+1$.
        On factorise : $=\frac{k(k+1)+2(k+1)}{2}=\frac{(k+1)(k+2)}{2}$ ;
        formule vraie pour $k+1$. %\qedhere
\end{itemize}

\begin{exercice}[Induction et recherche dichotomique]
\leavevmode\par
\textbf{1.} Commentez chaque étape du pseudo-code suivant :
\begin{lstlisting}[mathescape=true]
RechercheDicho(A[0..n-1], x)
    gauche ← 0, droite ← n-1
    Tant que gauche ≤ droite faire
        m ← ⌊(gauche+droite)/2⌋
        Si A[m] = x alors retourner m
        Sinon si A[m] < x alors gauche ← m+1
                       sinon droite ← m-1
    retourner -1                  // absent
\end{lstlisting}
\textbf{2.} Montrez par induction forte sur la taille du sous‑tableau que
l’algorithme renvoie toujours l’indice de \(x\) s’il existe, et $-1$ sinon.
\end{exercice}

% **********************************************************************
% ======================================================================
\section{Relations de récurrence}
% ----------------------------------------------------------------------
%  Inspiré de : Cormen et al., *Introduction to Algorithms*, chap. 4 ;
%               Kleinberg–Tardos, *Algorithm Design*, §2.1 ;
%               Sedgewick–Wayne, *Algorithms*, §2.3.
% ======================================================================

%\subsection{Pourquoi des récurrences ?}
Dès qu’un algorithme se définit \emph{sur lui-même} — via un appel récursif ou
une division en sous-problèmes — son coût obéit à une \textbf{relation de
récurrence}.  Résoudre cette équation, c’est transformer une description
\emph{locale} (« coût d’un appel ») en une formule \emph{globale}
\(T(n)\) qui vaut pour toute taille d’entrée.

\subsection{Méthodes classiques de résolution}

\begin{description}
  \item[Déroulement (ou \textit{iteration method})]
        On \emph{déplie} la récurrence plusieurs fois ; un motif apparaît
        (somme arithmétique, géométrique…), que l’on additionne jusqu’à
        atteindre le cas de base \(T(1)\).

  \item[Substitution]\phantom{a}\\
        \begin{enumerate}
         \item Formuler une \emph{conjecture} raisonnable $f(n)$.
         \item Prouver par induction (souvent forte) que $T(n)\le f(n)$ ou
           $T(n)=f(n)$.
        \end{enumerate}


  \item[Arbre de récursion]
        On dessine les appels comme un arbre : chaque niveau contient le coût
        cumulé des sous-problèmes.  La somme des niveaux donne $T(n)$ et
        révèle visuellement où se concentre la dépense (racine, feuilles, partout…).

  \item[Théorème maître (\textbf{Master Theorem})]
        Applicable aux récurrences $T(n)=aT(n/b)+f(n)$ avec $a\ge1$, $b>1$.
        Trois cas ; la comparaison entre $f(n)$ et
        \(n^{\log_b a}\) décide du résultat (voir tableau en annexe A).
\end{description}

% ----------------------------------------------------------------------
\subsection{Exemple — Coût de la recherche dichotomique}

\paragraph{Récurrence}
\[
T(n)=T\!\bigl(\tfrac{n}{2}\bigr)+1,\qquad T(1)=1.
\]

\begin{enumerate}[label=\alph*)]
  \item \textbf{Déroulement}
        \[
        \begin{aligned}
        T(n)&=T\!\bigl(\tfrac{n}{2}\bigr)+1\\
             &=T\!\bigl(\tfrac{n}{4}\bigr)+1+1\\
             &\;\;\vdots\\
             &=T(1)+\underbrace{1+\cdots+1}_{k\text{ fois}}
        \end{aligned}
        \]
        où \(n/2^{k}=1\Rightarrow k=\log_2 n\).
        \(\Rightarrow T(n)=1+\log_2 n\in\Theta(\log n).\)

  \item \textbf{Substitution rapide}
        Hypothèse : \(T(n)\le c\log n\).
        \(T(n)=T(n/2)+1\le c\log(n/2)+1
               =c(\log n-1)+1\le c\log n\) dès que \(c\ge1\).

  \item \textbf{Arbre de récursion}
        Chaque niveau coûte exactement 1 comparaison ; il y a
        \(\log_2 n+1\) niveaux $\Rightarrow$ même résultat.

  \item \textbf{Théorème maître}
        $a=1$, $b=2$, $f(n)=1$, \(n^{\log_b a}=1\).
        Cas 2 $\Rightarrow$ \(T(n)=\Theta(\log n)\).
\end{enumerate}

\begin{exercice}[Discussion]
Expliquez en une phrase pourquoi \(\log n\) apparaît inévitablement lorsqu’un
algorithme divise la taille de l’entrée par 2 à chaque étape.
\end{exercice}

% ----------------------------------------------------------------------
\begin{exercice}[Tri fusion]
\textbf{Algorithme}
\begin{lstlisting}
MergeSort(A[0..n-1]):
    si n ≤ 1 retourner
    diviser A en moities de taille n/2
    MergeSort(moitie_gauche)
    MergeSort(moitie_droite)
    fusionner les deux moities en temps lineaire n
\end{lstlisting}

\begin{enumerate}[label=\alph*)]
  \item Établissez la récurrence numérique $T(n)$ (pensez : 2 sous-listes +
        fusion).
  \item Résolvez $T(n)$ par déroulement, substitution et théorème maître.
  \item Comparez brièvement la lisibilité des trois démarches.
\end{enumerate}
\end{exercice}

% ----------------------------------------------------------------------
\begin{exercice}[Tri par insertion]
\textbf{Algorithme}
\begin{lstlisting}
InsertionSort(A[0..n-1]):
    pour i ← 1 a n-1:
        cle ← A[i]; j ← i-1
        tant que j ≥ 0 et A[j] > cle:
            A[j+1] ← A[j]; j ← j-1
        A[j+1] ← cle
\end{lstlisting}

\begin{enumerate}[label=\alph*)]
  \item Modélisez le \emph{pire coût} $T(n)$ par une relation de récurrence
        (indice : le $i$-ième passage décale au plus $i-1$ éléments).
  \item Résolvez-la par déroulement pour obtenir l’ordre de grandeur
        \(\Theta(n^{2})\).
  \item En quoi le \emph{meilleur cas} diffère-t-il ? (réponse : $T(n)=\Theta(n)$)
\end{enumerate}
\end{exercice}


% **********************************************************************
\section{Notation asymptotique}

% **********************************************************************
\subsection{$\bigO$, $\Omega$, $\Theta$ — définitions formelles}

Lorsque nous évaluons un algorithme, nous voulons savoir comment son
temps d’exécution ou sa consommation mémoire \emph{croissent} avec la taille
de l’entrée $n$.  Compter le nombre \emph{exact} d’opérations
($3n^{2}+7n-4$, par exemple) serait trop précis : les constantes $3$, $7$ ou
$4$ dépendent du langage, du processeur, parfois même du compilateur.
Ce qui nous intéresse vraiment, c’est l’\textbf{ordre de grandeur} :
le coût se comporte-t-il comme une droite ($n$), une parabole ($n^{2}$) ou,
pire, une exponentielle ($2^{n}$) ?

Les notations $\bigO$, $\Omega$ et $\Theta$ forment un langage standard pour
exprimer ces ordres de grandeur en « oubliant » les détails machines et les
constantes multiplicatives.  Elles permettent :
\begin{itemize}
  \item de comparer rapidement deux algorithmes : $\Theta(n\log n)$ battra
        toujours $\Theta(n^{2})$ pour des entrées suffisamment grandes ;
  \item d’estimer si un problème reste solvable quand l’entrée passe
        de quelques milliers à plusieurs millions d’éléments ;
  \item de raisonner sur la faisabilité \emph{avant} d’écrire la moindre
        ligne de code.
\end{itemize}
En pratique, la notation asymptotique est donc la \emph{boussole} qui guide
nos choix d’algorithmes et d’optimisations.


\begin{description}
  \item[Borne supérieure $\bigO$ : ]
        \(f(n)\in\bigO\bigl(g(n)\bigr)\)
        \(\Longleftrightarrow\) \(\exists\,C>0,\;\exists\,n_{0}\in\mathbb N,\;
        \forall n\ge n_{0},\;f(n)\le C\,g(n).\)

  \item[Borne inférieure $\Omega$ : ]
        \(f(n)\in\Omega\bigl(g(n)\bigr)\)
        \(\Longleftrightarrow\) \(\exists\,c>0,\;\exists\,n_{0},\;
        \forall n\ge n_{0},\;f(n)\ge c\,g(n).\)

  \item[Même ordre $\Theta$ : ]
        \(f(n)\in\Theta\bigl(g(n)\bigr)\)
        \(\Longleftrightarrow\) \(f\in\bigO(g)\) \emph{et}
        \(f\in\Omega(g)\).
        Autrement dit, $f$ est \emph{à la fois} bornée au-dessus et au-dessous
        par $g$ à multiplicateur constant près.

\end{description}

\vspace{0.4em}
\noindent
\textbf{Remarque pratique.}  Lorsqu’on écrit qu’un algorithme tourne en
$\bigO(n\log n)$, on déclare qu’il existe \emph{une} constante $C$ telle que,
pour une entrée assez grande, son temps d’exécution ne dépasse jamais
$C\,n\log n$.  Les détails machine-dépendants (vitesse CPU, langage, etc.)
sont absorbés dans ce $C$.

% ----------------------------------------------------------------------

\begin{exercice}[Comparer des croissances]
Classez les fonctions suivantes du plus lent au plus rapide :
$\log n$, $n^{0.5}$, $n\log n$, $2^{n}$, $n!$, $n^{2}$.
\end{exercice}

% **********************************************************************


% **********************************************************************
\section{Probabilités élémentaires pour l’analyse d’algorithmes}

\subsection{Pourquoi la probabilité en algorithmique ?}
Nombre d’algorithmes modernes (Quicksort randomisé, algorithmes de Monte-Carlo,
protocoles réseau) prennent des décisions au hasard ; leur \emph{coût} n’est
plus déterministe.  Les outils probabilistes permettent alors de :
\begin{itemize}
  \item décrire la « loi » des temps d’exécution possibles ;
  \item calculer une \textbf{espérance} (coût moyen) et parfois la probabilité
        de dépasser un seuil critique ;
  \item prouver qu’un algorithme est \emph{rapide avec grande probabilité}.
\end{itemize}

% ----------------------------------------------------------------------
\subsection{Vocabulaire et axiomes de base}
\begin{itemize}
  \item \textbf{Univers} $\Omega$ : ensemble de tous les résultats possibles
        d’une expérience aléatoire (ex. lancer un dé : $\{1,\dots,6\}$).
  \item \textbf{Événement} $A\subseteq\Omega$ : sous-ensemble d’issues.
  \item \textbf{Probabilité} \(\mathbb{P}[A]\) : fonction telle que
        \(\mathbb{P}\bigl[\Omega\bigr]=1\) et, si $A,B$ disjoints,
        \(\mathbb{P}[A\cup B]=\mathbb{P}[A]+\mathbb{P}[B]\).
\end{itemize}

\subsection{Espérance et variables indicatrices}
\begin{description}
  \item[Variable aléatoire] fonction $X:\Omega\to\mathbb{R}$ (ex. valeur
        obtenue sur un dé).
  \item[Espérance] \(\displaystyle \mathbb{E}[X]=\sum_{\omega\in\Omega}
        X(\omega)\,\mathbb{P}[\{\omega\}].\)
  \item[Linéarité] pour \emph{toutes} $X,Y$ :
        \(\mathbb{E}[X+Y]=\mathbb{E}[X]+\mathbb{E}[Y]\) même si $X$ et $Y$
        sont dépendantes.
  \item[Indicateur] \(\mathbf{1}_{A}(\omega)=1\) si $\omega\in A$, sinon 0.
        Utile pour compter : si $X=\sum_i \mathbf{1}_{A_i}$ alors
        \(\mathbb{E}[X]=\sum_i \mathbb{P}[A_i]\).
\end{description}

\subsection{Petit exemple classique}
Combien de lancers de pièce faut-il \emph{en moyenne} avant d’obtenir pile ?

\[
\mathbb{E}[X] = p\cdot1 + (1-p)p\cdot2 + (1-p)^{2}p\cdot3+\dots
              = \frac{1}{p},
\quad p=\tfrac12 \;\Rightarrow\; \mathbb{E}[X]=2.
\]

\subsection{Application : coût moyen d’une boucle aléatoire}
Supposons une boucle qui se répète tant qu’un événement de probabilité
\(p=\tfrac12\) ne se produit pas ; le travail à chaque itération vaut
\(\alpha\).  Son coût moyen est
\(\mathbb{E}[T]=\alpha\,\mathbb{E}[X]=2\alpha\).

\begin{exercice}[Taille d’un préfixe aléatoire]
On parcourt un tableau jusqu’à rencontrer la première valeur négative.
Chaque cellule est négative avec probabilité $q=\tfrac14$, indépendamment
des autres.  Calculez l’espérance du nombre de lectures effectuées.
\end{exercice}

\begin{reflexion}
Pourquoi la linéarité de l’espérance est-elle particulièrement précieuse
quand les variables sont dépendantes ?  Donne un exemple tiré d’un
algorithme où cette propriété simplifie drastiquement le calcul.
\end{reflexion}

% **********************************************************************
\section*{À retenir}

\begin{itemize}
  \item Sommes et logarithmes : les briques de base pour compter.
  \item Induction : la preuve standard pour algorithmes et formules.
  \item Récurrences : quatre outils (déroulement, substitution, arbre, maître).
  \item Notation asymptotique : comparer les croissances sans se perdre
        dans les constantes.
  \item Espérance : mesurer le coût moyen des algorithmes randomisés.
\end{itemize}
% **********************************************************************
% Liste de sept exercices progressifs (sans corrigés)
% **********************************************************************

\section{Exercices}
\begin{exercice}[Somme arithmétique]
Calculez, par induction simple, la valeur de la somme
\(\displaystyle S(n)=\sum_{i=1}^{n}(2i-1)\)
et vérifiez que \(S(n)\) est égale à \(n^{2}\).  %&#8203;:contentReference[oaicite:0]{index=0}
\end{exercice}

\begin{exercice}[Manipuler les logarithmes]
Montrez que pour tout \(n\ge1\),
\[
\log_2(n!) \;=\; \Theta\!\bigl(n\log n\bigr).
\]
(On pourra encadrer la somme \(\sum_{k=1}^{n}\log_2 k\) par une intégrale.)  %&#8203;:contentReference[oaicite:1]{index=1}
\end{exercice}

\begin{exercice}[Induction forte et nombres de Fibonacci]
Prouvez, par induction forte, que pour tout \(n\ge 0\),
le \(n\)-ième nombre de Fibonacci vérifie
\(F_{n}\le 2^{n}\).  %&#8203;:contentReference[oaicite:2]{index=2}
\end{exercice}

\begin{exercice}[Récurrence linéaire]
Soit \(T(n)=3T(n-1)+2\) avec \(T(0)=4\).
\begin{enumerate}[label=\alph*)]
  \item Trouvez la forme fermée de \(T(n)\) par déroulement.
  \item Donnez l’ordre de grandeur asymptotique de \(T(n)\).
\end{enumerate} %&#8203;:contentReference[oaicite:3]{index=3}
\end{exercice}

\begin{exercice}[Complexité espérée d’un algorithme aléatoire]
Un algorithme répète une opération coûtant une unité tant que
un événement de probabilité \(p=\tfrac13\) ne se produit pas.
Quel est le coût moyen de l’algorithme ?  %&#8203;:contentReference[oaicite:4]{index=4}
\end{exercice}

\begin{exercice}[MergeSort — récurrence à résoudre]
Considérez l’algorithme \texttt{MergeSort} :
\[
T(n)=2\,T\!\bigl(\tfrac{n}{2}\bigr)+n,\qquad T(1)=1.
\]
\begin{enumerate}[label=\alph*)]
  \item Résolvez la récurrence par la méthode du \emph{déroulement}.
  \item Vérifiez votre résultat avec le théorème maître.
\end{enumerate} %&#8203;:contentReference[oaicite:5]{index=5}
\end{exercice}

\begin{exercice}[InsertionSort — formuler la récurrence]
\textbf{Algorithme}\\[-1.1em]
\begin{lstlisting}
InsertionSort(A[0..n-1]):
    pour i ← 1 a n-1:
        cle ← A[i]; j ← i-1
        tant que j ≥ 0 et A[j] > cle:
            A[j+1] ← A[j]; j ← j-1
        A[j+1] ← cle
\end{lstlisting}

\begin{enumerate}[label=\alph*)]
  \item Établissez la récurrence du \emph{pire cas} pour le coût $T(n)$.
  \item Déduisez le classement de ce coût en notation~$\bigO$.
\end{enumerate} %&#8203;:contentReference[oaicite:6]{index=6}
\end{exercice}

% ----------------------------------------------------------------------
% Bibliographie locale (le .bib global suffit en vrai, mais on note ici)
% ----------------------------------------------------------------------
\begin{thebibliography}{9}
\bibitem{Cormen2022} T.~Cormen, C.~Leiserson, R.~Rivest, and C.~Stein, \emph{Introduction to Algorithms}, 4th~ed., MIT Press, 2022.
\end{thebibliography}

% ----------------------------------------------------------------------
% Fin du chapitre
% ----------------------------------------------------------------------
