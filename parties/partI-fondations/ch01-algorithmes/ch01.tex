% ======================================================================
%  Chapitre 1 – Algorithmes et Programmes (Partie I : Fondations)
%  Public : étudiantes et étudiants universitaires n’ayant jamais suivi de
%           cours d’algorithmique.
%  Objectif : poser les bases avec un langage clair, des exemples concrets
%             et plusieurs mini‑exercices d’appropriation.
% ======================================================================

\chapter{Algorithmes et programmes : notions fondamentales}

% ----------------------------------------------------------------------
% Mise en bouche (citation / épigraphe)
% ----------------------------------------------------------------------
\begin{flushright}\small
« Un programme n’est qu’un algorithme écrit dans une langue que la machine
comprend. »\\[-0.2em]
— \textit{Donald E. Knuth}
\end{flushright}

Un \textbf{algorithme} est à l’informatique ce qu’une recette est à la cuisine : un
ensemble d’étapes précises qui transforment des \emph{ingrédients} (les données
d’entrée) en un \emph{plat fini} (le résultat). Derrière chaque application —
banque en ligne, réseau social, GPS, moteur de recherche — se cachent des
algorithmes. Étudier leur conception permet :
\begin{enumerate}
  \item de \textbf{formaliser} la résolution de problèmes ;              % logique
  \item de \textbf{prouver} qu’une méthode fonctionne toujours ;        % correction
  \item d’estimer le \textbf{temps} et la \textbf{mémoire} nécessaires ; % performance
  \item d’améliorer continuellement l’efficacité des logiciels.          % optimisation
\end{enumerate}

\section*{Objectifs du chapitre}
\begin{itemize}[label=\small$\blacktriangleright$]
  \item Distinguer clairement \emph{algorithme}, \emph{programme} et \emph{implémentation}.
  \item Savoir décrire un algorithme simple en pseudo‑code.
  \item Comprendre pourquoi il est utile de mesurer le temps et la mémoire.
  \item S’exercer sur des problèmes très élémentaires (tri, PGCD, factorielle).
\end{itemize}
\vspace{0.5em}

% **********************************************************************
%\section{Pourquoi parler d’algorithmes ?}


\begin{reflexion}
Donnez des activités quotidiennes (hors informatique) qui suivent déjà un
algorithme implicite. Que se passerait‑il si les étapes étaient exécutées dans
le désordre ?
\end{reflexion}

% **********************************************************************
\section{Vocabulaire de base}

% **********************************************************************
\subsection{Algorithme }

Un \textbf{algorithme} est une suite finie d’instructions respectant quatre
critères essentiels :
\begin{description}
  \item[Finitude] chaque exécution comporte un nombre limité d’étapes, et
        l’algorithme se termine toujours après un nombre fini d’opérations.
  \item[Déterminisme] à chaque étape, l’instruction suivante est parfaitement
        définie, sans ambigüité ni hasard non maîtrisé.
  \item[Non-ambiguïté] chaque action (affectation, comparaison…) est formulée
        de façon à ne pouvoir être interprétée que d’une seule manière.
  \item[Entrées / sorties] l’algorithme reçoit une \emph{entrée} bien définie
        et produit une \emph{sortie} (résultat) correspondant exactement à la
        spécification.
\end{description}

Ces principes garantissent que toute personne ou machine exécutant l’algorithme
obtiendra toujours le même résultat en un temps fini.

% ----------------------------------------------------------------------

Voici un exemple de pseudo-code illustrant les points essentielles d'un algorithme :

\begin{lstlisting}[mathescape=true]
fonction linearSearch(A[0..n-1], key):
    // A : tableau d'entiers de taille n
    // key : valeur recherchee
    pour i de 0 a n-1 faire          // parcours sequentiel
        si A[i] == key alors         // test d'egalite
            retourner i              // on renvoie l'indice trouve
        fin si
    fin pour
    retourner -1                     // indicateur d'absence
\end{lstlisting}

\noindent
\begin{itemize}
 \item La variable `i` parcourt chaque case de `A`.
 \item La condition `A[i] == key` est une opération élémentaire.
 \item En pire cas, toutes les \(n\) cases sont examinées.
 \item L’usage de mots-clés en français (`pour`, `si`, `fin si`) rend le code
  indépendant d’un langage particulier et très clair pour un débutant.
\end{itemize}

\subsection{Programme}
Un \emph{programme} est l’implémentation concrète d’un algorithme dans un
\textbf{langage formel} que l’ordinateur peut comprendre et exécuter.
Le code source décrit :
\begin{itemize}
  \item la \textbf{structure des données} (types, variables) ;
  \item le \textbf{flot de contrôle} (fonctions, boucles, conditions) ;
  \item les opérations d’\textbf{entrée/sortie} (lecture, affichage, fichiers…).
\end{itemize}
Le traducteur peut être :
\begin{itemize}
  \item un \emph{compilateur} (C++, Java → bytecode/masque binaire) ;
  \item un \emph{interpréteur} (Python, commandes successives à la volée).
\end{itemize}
Le résultat final est un exécutable ou un script prêt à tourner sur la machine.

% ----------------------------------------------------------------------
% ----------------------------------------------------------------------
\paragraph{Mise en œuvre pratique}
Pour illustrer la traduction d’un algorithme en programme, voici trois
implémentations de la recherche linéaire (pseudo-code précédent) en
C++, Python et Java :


\paragraph{C++ (linear\_search.cpp)}
\begin{lstlisting}[language=C++,caption={linear\_search.cpp}]
#include <vector>
#include <iostream>

// Renvoie l'indice de key dans A, ou -1 si absent
int linearSearch(const std::vector<int>& A, int key) {
    for (size_t i = 0; i < A.size(); ++i) {
        if (A[i] == key) return static_cast<int>(i);
    }
    return -1;
}

int main() {
    std::vector<int> A = {5, 3, 8, 4, 2};
    int key = 4;
    int idx = linearSearch(A, key);
    if (idx >= 0)
        std::cout << "Trouve a l'indice " << idx << "\n";
    else
        std::cout << "Non trouve\n";
    return 0;
}
\end{lstlisting}

\paragraph{Python (linear\_search.py)}
\begin{lstlisting}[language=Python,caption={linear\_search.py}]
# Renvoie l'indice de key dans A, ou -1 si absent
def linear_search(A, key):
    for i, v in enumerate(A):
        if v == key:
            return i
    return -1

if __name__ == "__main__":
    A = [5, 3, 8, 4, 2]
    key = 4
    idx = linear_search(A, key)
    if idx != -1:
        print(f"Trouve a l'indice {idx}")
    else:
        print("Non trouve")
\end{lstlisting}

\paragraph{Java (LinearSearch.java)}
\begin{lstlisting}[language=Java,caption={LinearSearch.java}]
public class LinearSearch {
    // Renvoie l'indice de key dans A, ou -1 si absent
    public static int linearSearch(int[] A, int key) {
        for (int i = 0; i < A.length; i++) {
            if (A[i] == key) return i;
        }
        return -1;
    }

    public static void main(String[] args) {
        int[] A = {5, 3, 8, 4, 2};
        int key = 4;
        int idx = linearSearch(A, key);
        if (idx >= 0)
            System.out.println("Trouve a l'indice " + idx);
        else
            System.out.println("Non trouve");
    }
}
\end{lstlisting}

\subsection{Spécification et implémentation}
% ----------------------------------------------------------------------

Avant d’écrire la moindre ligne de code, il est indispensable de
séparer clairement deux niveaux de description : la \textbf{spécification},
qui répond à la question \og que doit faire le programme ? \fg{}, et
l’\textbf{implémentation}, qui détaille \og comment il va s’y prendre ? \fg{}.
Cette distinction assure une conception rigoureuse, facilite la preuve de
correction et simplifie la maintenance.


Par exemple, « trier une liste de nombres entiers en ordre croissant. » est une spécification tandis qu'\og utiliser le tri par insertion pour trier\fg est une implémentation.


\begin{exercice}[Spécification ou implémentation ?]
Pour chaque phrase, dire si \emph{S} (spécification) ou \emph{I} (implémentation)
et justifier votre choix :
\begin{enumerate}[label=\alph*)]
  \item \og Trouver le plus grand nombre dans une liste. \fg{}
  \item \og Parcourir la liste et mémoriser la valeur maximale rencontrée. \fg{}
  \item \og Chiffrer un message selon la clé fournie. \fg{}
  \item \og Pour chaque caractère, appliquer un décalage de trois positions dans
        l’alphabet (chiffrement de César). \fg{}
\end{enumerate}
\end{exercice}

% **********************************************************************
\section{Exemple : le plus grand commun diviseur}

Nous allons calculer le \textbf{PGCD} de deux entiers grâce à l’algorithme
millénaire d’Euclide. Pourquoi ce choix ? Il est court, toujours correct et ses
performances se mesurent facilement.

\subsection{Pseudo‑code}
\begin{lstlisting}[caption={Algorithme d'Euclide (version itérative)}]
Entree : deux entiers strictement positifs a, b (a ≥ b)
Sortie : g = pgcd(a, b)

g ← a, h ← b
Tant que h ≠ 0 faire
    r ← g mod h   // reste de la division de g par h
    g ← h
    h ← r
Retourner g
\end{lstlisting}

\begin{itemize}
  \item \texttt{g} contient la valeur courante du PGCD présumé.
  \item Tant que le reste \texttt{h} n’est pas nul, on poursuit la division.
  \item Quand \texttt{h} vaut 0, la dernière valeur non nulle de \texttt{g} est
        le PGCD.
\end{itemize}

\subsection{Implémentation C++}
\begin{lstlisting}[language=C++,caption=euclid.cpp]
#include <iostream>
#include <cstdint>

std::uint64_t pgcd(std::uint64_t a, std::uint64_t b) {
    while (b != 0) {
        auto r = a % b;
        a = b;
        b = r;
    }
    return a;
}

int main() {
    std::uint64_t x, y;
    std::cin >> x >> y;
    std::cout << pgcd(x, y) << "\n";
}
\end{lstlisting}

\subsection{Et si on mesurait le coût ?}
À chaque tour de boucle, on fait une division euclidienne. Le mathématicien
Gabriel Lamé a montré qu’il suffit de \(5\log_{10}(b)\) itérations au plus
lorsque \(b\) est le plus petit des deux nombres. Le temps d’exécution est donc
proportionnel à \(\log \min(a, b)\), noté $\bigO(\log n)$.

\begin{reflexion}
Essayez manuellement l’algorithme pour \(a = 48,\, b = 18\). Combien
d’itérations obtenez‑vous ?
\end{reflexion}

% **********************************************************************
\section{Premiers pas vers l’analyse de complexité}

\subsection{Temps (nombre d’opérations)}
On compte \emph{combien} d’étapes élémentaires (addition, comparaison, etc.)
le programme effectue en fonction de la taille de l’entrée (\(n\)). Cette mesure
s’appelle la \emph{complexité temporelle}. On utilise souvent la notation
$\bigO$ (\og grand O \fg) pour donner une borne supérieure grossière.

\subsection{Mémoire (espace occupé)}
Même idée : combien de cases supplémentaires devons‑nous réserver ? Pour
l’algorithme d’Euclide, on se contente de trois variables entières ; l’espace est
\emph{constant} ($\bigO(1)$).

\begin{exercice}[Itératif vs récursif]
\textbf{a)} Donnez une version \emph{récursive} du calcul de la factorielle
\(n!\).\\
\textbf{b)} Donnez la version \emph{itérative}.\\
\textbf{c)} Comparez les deux en temps et en espace (pile d’appels).
\end{exercice}

% **********************************************************************
\section*{À retenir}
\begin{itemize}
  \item Un \textbf{algorithme} = recette finie et non ambiguë.
  \item Un \textbf{programme} = algorithme + langage + machine.
  \item Spécification (\og quoi ?\fg) \emph{$\neq$} implémentation (\og comment ?\fg).
  \item Avant d’optimiser, on mesure : temps $\bigO(\cdot)$ et mémoire.
\end{itemize}

\begin{tp}[De l’algorithme au programme]
\textbf{But :} créer une mini‑bibliothèque C++ nommée \texttt{arith} contenant
\texttt{pgcd}, \texttt{ppcm} et \texttt{factorielle}, puis en vérifier la
correction par des tests unitaires (Catch2).

\textbf{Étapes proposées :}
\begin{enumerate}
  \item Écrire un fichier d’en‑tête \texttt{arith.hpp} (déclarations).
  \item Implémenter dans \texttt{arith.cpp}.
  \item Configurer un \texttt{CMakeLists.txt} minimal pour la compilation.
  \item Rédiger des tests couvrant : cas triviaux (0, 1), cas usuels, cas
        \og limites \fg{} (valeurs proches de la capacité d’un \texttt{uint64\_t}).
  \item Mesurer les temps moyens d’exécution pour des entrées de plus en plus
        grandes à l’aide de \texttt{std::chrono}.
\end{enumerate}
\end{tp}

% **********************************************************************

% ----------------------------------------------------------------------
% Fin du chapitre
% ----------------------------------------------------------------------
